\begin{frontmatter}


\title{Nontrivial behavior of the fixed-point version of $2D$-chaotic maps}

\author[mymainaddress,mysecondaryaddress]{L. De Micco\corref{mycorrespondingauthor}}
\cortext[mycorrespondingauthor]{Corresponding author}
\ead{ldemicco@fi.mdp.edu.ar}

\author[mymainaddress]{M. Antonelli}

\author[mymainaddress,mysecondaryaddress]{H. A. Larrondo}

\address[mymainaddress]{ICYTE (Instituto de Investigaciones Cient\'ificas y Tecnol\'ogicas en Electr\'onica)\\ Facultad de Ingenier\'ia, Universidad Nacional de Mar del Plata\\Juan B. Justo 4302, Mar del Plata\\Buenos Aires, Argentina.}
\address[mysecondaryaddress]{ CONICET (Consejo Nacional de Investigaciones Cient\'ificas y T\'ecnicas)}

\begin{abstract}
This paper deals with a family of interesting $2D$-quadratic maps
 proposed by Sprott, in his seminal paper \cite{Sprott1993},
related to ``chaotic art".  Only results for the analytical representation of these
maps have been published in the open literature. Our main interest about these maps is
they may be used to generate a novel encryption system, because
they present multiple chaotic attractors depending on the selected
point in the parameter's space. Consequently the objective of
this paper is to extend the analysis to the digital version, to make
possible  the hardware implementation in a digital medium, like field programmable gate arrays (FPGA) in fixed-point arithmetic.
Our main contributions are: (a) the study of the domains of attraction in fixed-point arithmetic; (b)  the determination of the \textsl{threshold} of the bus width that preserves the integrity of the domain of
attraction and (c) the comparison between two quantifiers based on respective probability distribution functions and  the well known maximum Lyapunov exponent (MLE) to detect the above mentioned threshold. 
\end{abstract}

\begin{keyword}
2D-quadratic map's digitalization\sep randomness quantifier \sep fractional length \sep chaotic map's degradation
\end{keyword}

\end{frontmatter}