\listfiles
\documentclass[manuscript, review, screen]{acmart}
%\setcitestyle{super,sort&compress}
\citestyle{acmauthoryear}
\usepackage{booktabs} % For formal tables

\usepackage[ruled]{algorithm2e} % For algorithms
\usepackage{enumitem}

\graphicspath{{Images/}}
\usepackage{graphicx}

% Metadata Information

\acmYear{2017}
\acmMonth{3}

%\acmBadgeL[http://ctuning.org/ae/ppopp2016.html]{ae-logo}
%\acmBadgeR[http://ctuning.org/ae/ppopp2016.html]{ae-logo}

% Copyright
%\setcopyright{acmcopyright}
%\setcopyright{acmlicensed}
%\setcopyright{rightsretained}
%\setcopyright{usgov}
%\setcopyright{usgovmixed}
%\setcopyright{cagov}
%\setcopyright{cagovmixed}

% DOI
\acmDOI{0000001.0000001}


% Document starts
\begin{document}
% Title portion
\title{Nontrivial behavior of the fixed-point version of $2D$-chaotic maps} 
% \titlenote{This is a titlenote}
% \subtitle{This is a subtitle}
% \subtitlenote{Subtitle note}
\author{L. De Micco}

\email{ldemicco@fi.mdp.edu.ar}
\author{M. Antonelli}
\author{H. A. Larrondo}
\affiliation{%
  \institution{ICYTE (Instituto de Investigaciones Cient\'ificas y Tecnol\'ogicas en Electr\'onica)}
  \department{School of Engineering}
}
\affiliation{%
  \institution{CONICET ( Consejo  Nacional de
Investigaciones Cient\'{\i}ficas y T\'ecnicas)}}
%\author{M. Antonelli}
%\affiliation{%
%  \institution{ICYTE (Instituto de Investigaciones Cient\'ificas y Tecnol\'ogicas en Electr\'onica)}
%  \department{School of Engineering}
%}
%\author{H. A. Larrondo}
%\affiliation{%
%  \institution{ICYTE (Instituto de Investigaciones Cient\'ificas y Tecnol\'ogicas en Electr\'onica)}
%  \department{School of Engineering}
%}
%\affiliation{%
%  \institution{CONICET}}
%  

\begin{abstract}
This paper deals with a family of interesting $2D$-quadratic maps
 proposed by Sprott, in his seminal paper \cite{Sprott1993},
related to ``chaotic art".  Only results for the analytical representation of these
maps have been published in the open literature. Our main interest about these maps is their great potential in the use of digital electronic applications, because
they present multiple chaotic attractors depending on the selected
point in the parameter's space. Consequently the objective of
this paper is to extend the analysis to the digital version, to make
possible  the hardware implementation in a digital medium, like field programmable gate arrays (FPGA) in fixed-point arithmetic.
Our main contributions are: (a) the study of the domains of attraction in fixed-point arithmetic; (b)  the determination of the \textsl{threshold} of the bus width that preserves the integrity of the domain of
attraction and (c) the comparison between two quantifiers based on respective probability distribution functions and  the well known maximum Lyapunov exponent (MLE) to detect the above mentioned threshold. 
\end{abstract}



\keywords{2D-quadratic map's digitalization, randomness quantifier, fractional length, chaotic map's degradation}


%\thanks{This work is supported by the National Science Foundation,
%  under grant CNS-0435060, grant CCR-0325197 and grant EN-CS-0329609.
%
%  Author's addresses: G. Zhou, Computer Science Department, College of
%  William and Mary; Y. Wu {and} J. A. Stankovic, Computer Science
%  Department, University of Virginia; T. Yan, Eaton Innovation Center;
%  T. He, Computer Science Department, University of Minnesota; C.
%  Huang, Google; T. F. Abdelzaher, (Current address) NASA Ames
%  Research Center, Moffett Field, California 94035.}


\maketitle

%


\input{Introduction}

\input{ProblemStatement}

\input{ChaoticSystemUnderStudy}

\input{AnalysisTools}

\input{HardwareDigitalSimulation}

\input{Results}

\input{Conclusions}


\section*{Acknowledgment}
This work was partially  supported  by  the  Consejo  Nacional de
Investigaciones Cient\'{\i}ficas y T\'ecnicas (CONICET), Argentina
(PIP 112-201101-00840), ANPCyT (PICT-2013-2066), UNMDP and the International Centre for Theoretical Physics (ICTP) Associateship Scheme.\\

% Bibliography
\bibliographystyle{ACM-Reference-Format}
\bibliography{BibPaperSprott}

\end{document}
